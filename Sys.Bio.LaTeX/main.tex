\documentclass[webpdf,PV,mynatbib,surname,CE,MSC]{SYS-PV}

\PVskip=-30.5pt

\titleskip=50pt

\volname{}

\jvolume{00}

\jshort{syv}

\jvol{}

\jissue{0}

\access{Advance Access publication Xxx XX, XXXX}

\def\querybox#1{\protect\fboxsep=0pt\colorbox{yellow}{#1}}

\cename{XX}

\mstype{Points of View}

\pubyear{2015}

\copyyear{2015}

\artid{xxx}

\renewcommand\theequation{\arabic{equation}}
\setcounter{secnumdepth}{0}

\begin{document}

\title[AUTHOR ET AL.---ARTICLE TITLE FOR POINTS OF VIEW]{Article Title for Points of View}

\author[]{A\textsc{UTHOR} \surname{N\textsc{AME}}$^{1,\ast}$,
A\textsc{UTHOR} \surname{N\textsc{AME}}$^{2,3,\S}$,
A\textsc{UTHOR} \surname{N\textsc{AME}}$^{4}$, A\textsc{UTHOR}
\surname{N\textsc{AME}}$^{5}$, A\textsc{UTHOR}
\surname{N\textsc{AME}}$^{1}$, \textsc{AND}
A\textsc{UTHOR}~\surname{N\textsc{AME}-A\textsc{UTHOR}}$^{1}$}

\address{$^1$Molecular Ecology Group, Institute of Ecology, University of Innsbruck, Technikerstr. 25, 6020 Innsbruck, Austria
$^{2}$Forest Pathology and Dendrology, Institute of Integrative Biology, ETHZ, Universit\"{a}tstr. 16, 8092 Zurich, Switzerland
$^{3}$Animal and Plant Health Unit, European Food Safety Authority, via Carlo Magno 1a, 43126 Parma, Italy
$^{4}$Department of Entomology, Natural History Museum Vienna, Burgring 7, 1010 Vienna, Austria
$^{5}$Institute of Applied Statistics and Computing, University of Natural Resources and Life Sciences, Peter Jordan-Str. 82, 1180 Vienna, Austria\\
$^{\ast}$Correspondence to be sent to: Molecular Ecology Group, Institute of Ecology, University
of Innsbruck, Technikerstr. 25, 6020 Innsbruck, Austria;\newline
E-mail:~florian.m.steiner@uibk.ac.at\\
$^{\S}$The positions and opinions presented in this article are those of the authors alone and are
not intended to represent the views or scientific works of EFSA.}

\history{Received 00 Xxxxxx 0000; reviews returned 00 Xxxxx 0000;
accepted 00 Xxxxx 0000}

\editor{Associate Editor: Xxxxx Xxxx}

\abstract{Current science evaluation still relies on citation
performance, despite criticisms of purely bibliometric research
assessments. Biological taxonomy suffers from a drain of knowledge
and manpower, with poor citation performance commonly held as one
reason for this impediment. But is there really such a citation
impediment in taxonomy? We compared the citation numbers of 306
taxonomic and 2291 nontaxonomic research articles (2009--2012) on
mosses, orchids, ciliates, ants, and snakes, using Web of Science
(WoS) and correcting for journal visibility. For three of the five
taxa, significant differences were absent in citation numbers
between taxonomic and nontaxonomic papers. This was also true for
all taxa combined, although taxonomic papers received more
citations than nontaxonomic ones. Our results show that, contrary
to common belief, taxonomic contributions do not generally reduce
a journal's citation performance and might even increase it. The
scope of many journals rarely featuring taxonomy would allow
editors to encourage a larger number of taxonomic submissions.
Moreover, between 1993 and 2012, taxonomic publications
accumulated faster than those from all biological fields. However,
less than half of the taxonomic studies were published in journals
in WoS. Thus, editors of highly visible journals inviting
taxonomic contributions could benefit from taxonomy's strong
momentum. The taxonomic output could increase even more than at
its current growth rate if: (i) taxonomists currently publishing
on other topics returned to taxonomy and (ii) nontaxonomists
identifying the need for taxonomic acts started publishing these,
possibly in collaboration with taxonomists. Finally, considering
the high number of taxonomic papers attracted by the journal
Zootaxa, we expect that the taxonomic community would indeed use
increased chances of publishing in WoS indexed journals. We
conclude that taxonomy's standing in the present citation-focused
scientific landscape could easily improve---if the community
becomes aware that there is no citation impediment in
taxonomy.[Animals; citations; impact factor; microorganisms;
plants; scientometrics; taxonomic impediment; taxonomy.]}

\maketitle


Biological taxonomy, the science of characterizing, classifying, and naming animate beings, is
essential in most basic and applied biosciences---and even beyond (\citealt{12Bortolus2008}). For
example, Australian wheat worth AUD 18 million was wasted for reasons of taxonomic confusion in
biosecurity control (\citealt{13Boykin2011}). Nevertheless, taxonomy is currently facing a
shortage in knowledge and manpower. This shortage blocks the desirable acceleration of describing
the remaining unknown species as well as progress in fighting the ever worsening biodiversity
crisis. Many reasons for this taxonomic impediment have been proposed, such as a limitation of job
opportunities for taxonomists (\citealt{3Agnarsson2007}; \citealt{20Ebach2011};
\citealt{33Lester2014}), a decreasing number of taxonomists (\citealt{25Gaston1992};
\citealt{35Mora2011}), insufficient propagation of taxonomic knowledge at universities
(\citealt{48Swiss2007}; \citealt{7Bilton2014}), and an advantage of nontaxonomic over taxonomic
proposals with funding agencies (\citealt{48Swiss2007}). Another reason proposed is that
taxonomists are not competitive enough in quantitative science evaluations in which standard
bibliometric measures are used (\citealt{31Krell2000}; \citealt{51Valdecasas2000};
\citealt{32Krell2002}; \citealt{20Ebach2011}; \citealt{34McDade2011}; \citealt{50Valdecasas2011};
\citealt{53Venu2011}; \citealt{6Benitez2014}; \citealt{7Bilton2014}; \citealt{19De2014};
\citealt{39Pyke2014}).

The reasons identified for taxonomists' poor performance according to standard bibliometrics
include: (i) Taxon authorities are not included in the reference sections of most publications
(\citealt{55Werner2006}; \citealt{47Sundberg2009}; \citealt{8Bininda-Emonds2011}). (ii) Taxonomy
is a slow field with long time lags between publication and citation of papers
(\citealt{32Krell2002}; \citealt{53Venu2011}). (iii) It is the editorial policy of most journals
with high impact factor (IF) to discourage taxonomic publications (\citealt{3Agnarsson2007};
\citealt{20Ebach2011}); various journals aiming to increase their IF have shifted their scope from
taxonomy to phylogenetics and molecular research (\citealt{43Shashank2014}). (iv) Taxonomic papers
have small immediate audiences (\citealt{20Ebach2011}) and receive few citations even when
published in high-quality journals (\citealt{51Valdecasas2000}). (v) Journals included in Web of
Science (WoS) that publish taxonomy tend to receive a low IF (\citealt{55Werner2006};
\citealt{53Venu2011}; \citealt{7Bilton2014}; \citealt{43Shashank2014}). (vi) Many journals
publishing taxonomy are not included in WoS and thus have no IF at all (\citealt{31Krell2000};
\citealt{9Boero2001}; \citealt{32Krell2002}; \citealt{48Swiss2007}; \citealt{10Boero2010};
\citealt{34McDade2011}; \citealt{6Benitez2014}; \citealt{43Shashank2014}).

\enlargethispage{2pt}

Various countermeasures have been proposed to overcome the taxonomic impediment, many of which
need policy measures, for example, providing more funding ear-marked to taxonomy within
universities, museums, and funding agencies (e.g., \citealt{18De2007}). Without denying the
importance of such measures, there have been suggestions for strategies manageable by scientists
themselves, aimed at improving the competitiveness and thus the standing of taxonomists. One
suggestion to achieve this within the existing system of science evaluation is to include
taxonomic authorities in the reference sections of publications in all instances
(\citealt{55Werner2006}; \citealt{8Bininda-Emonds2011}; \citealt{54Wagele2011}) or at least when
credit is due (\citealt{3Agnarsson2007}). Others have gone further and suggested alternative
evaluation methods (e.g., \citealt{34McDade2011}; \citealt{50Valdecasas2011};
\citealt{53Venu2011}; \citealt{41Schekman2013}; \citealt{39Pyke2014}). However, despite the
generally acknowledged criticism of the established bibliometric methods (\citealt{44Simons2008};
\citealt{1Adler2009}; \citealt{2Adler2009}; \citealt{37Patterson2009}; \citealt{14Brumback2012};
\citealt{23Eyre-Walker2013}; \citealt{24Foley2013}; \citealt{30Kaushal2013};
\citealt{41Schekman2013}), these are widely used for evaluating individuals, institutions, and
journals in research (\citealt{44Simons2008}; \citealt{52Vale2012}; \citealt{30Kaushal2013}).
Thus, at least in the shorter term, it seems difficult for the field of taxonomy to avoid
citation-based evaluations.

We ask whether citation-based evaluations indeed are a drawback for taxonomy. To our knowledge,
the citation performance of taxonomic papers has not been compared quantitatively with that of
nontaxonomic papers. In more detail, we address five questions. Does publishing taxonomy harm a
journal's citation performance? Is it within the possibilities of journal editors to influence
taxonomy's visibility? If more high-visibility journals opened their doors to taxonomic
publications, would taxonomy's productivity be sufficient for an increase in the number of
taxonomic papers in these journals? Can taxonomy be published by taxonomists only or by a larger
community? And finally, would the community use the chance to publish more taxonomic papers in
highly visible journals?\vs{-1}

\enlargethispage{-1pt}

\section*{P{\sc UBLISHING} T{\sc AXONOMY} D{\sc OES} N{\sc OT} G{\sc ENERALLY} H{\sc ARM
AND} M{\sc IGHT} E{\sc VEN} B{\sc OOST} J{\sc OURNALS}}

We present the results of a citation analysis on primary research articles on mosses, orchids,
ciliates, ants, and snakes as representatives of nonvascular plants, vascular plants,
heterotrophic microorganisms, invertebrates, and vertebrates, respectively. The five taxa were
randomly chosen among candidate taxa; candidate taxa needed to meet the criteria of (i) sufficient
taxonomic and nontaxonomic publications for sufficient sample sizes, and (ii) the trivial and the
scientific name applying to exactly the same taxon (e.g., every orchid belongs in the Orchidaceae,
and all species of the Orchidaceae are orchids). We were interested in the current situation and
thus chose the years 2009--2012; 2013 was not yet feasible because papers need some time to become
cited. For each of the five taxa, we selected the 10 journals included in WoS that published the
largest numbers of articles on the selected taxon in this period, totalling 47 journals (overlap
of three journals among taxa; Table~\ref{t1}; see Online Appendix 1 available as Supplementary
Material on Dryad at
\href{http://dx.doi.org/10.5061/dryad.3t761}{http://dx.doi.org/10.5061/dryad.3t761}, for the
protocols of database queries and manual content curation and Online Appendix 2 for the data
available as Supplementary Material on Dryad at
\href{http://dx.doi.org/10.5061/dryad.3t761}{http://dx.doi.org/10.5061/dryad.3t761}). We
classified the 2597 publications on the focal taxa according to what we term the factor Taxonomy,
that is, into taxonomic \hbox{($n=306$)} or nontaxonomic ($n=2291$). Papers were considered as
taxonomic if they included taxonomic acts at the genus to variety level, that is, not only just
descriptions of new taxa but also synonymizations, revivals from synonymy, and new combinations.
All these taxonomic acts represent relevant achievements by taxonomy and are needed to properly
assess biodiversity. We then analyzed the number of citations obtained by each publication as of
15 August 2014.

\begin{table*}[!p]%T1
\tableparts{\caption{The journals included in WoS publishing the largest numbers of publications
on mosses, orchids, ciliates, ants, and snakes (2009--2012) and their 2012 IF\label{t1}}}
{\tabcolsep=0pt\begin{tabular*}{\textwidth}{@{\extracolsep{\fill}}lclccccccccccc@{}}\toprule&\\
&&&&\multicolumn{2}{c}{All years} &\multicolumn{2}{c}{2009} &\multicolumn{2}{c}{2010}
&\multicolumn{2}{c}{2011} &\multicolumn{2}{c}{2012}
\down\\\cline{4-4}\cline{5-6}\cline{7-8}\cline{9-10}\cline{11-12}\cline{13-14}\up
Taxon& &Journal&IF 2012&Tax&Nontax&Tax&Nontax&Tax&Nontax&Tax&Nontax&Tax&Nontax \\\colrule&\\
Mosses&$\bullet$&Bryologist&0.98&25&58&9&16&4&18&6&16&6&8 \\
& $\bullet$& Cryptogamie Bryol.&1.04& 9& 49& 3& 17& 2& 6&  2& 9& 2&17 \\
~& ~& Environ. Pollut. & 3.73& 0& 28&0& 10&  0& 5& 0& 5& 0&8 \\
~& ~& Global Change Biol.& 6.91& 0& 23&  0& 7&  0& 1&  0& 6& 0&9 \\
~& $\bullet$& J. Bryol.& 1.35&18& 75& 4& 22& 4& 13& 9& 16&  1&24 \\
~& ($\bullet$)& New Phytol.& 6.74& 0& 26&  0& 8& 0& 2& 0& 8& 0&8 \\
~& $\bullet$& Nova Hedwigia& 0.81&  9& 43&  2& 4& 5& 19& 1& 11&  1&9 \\
~& ~& Oecologia & 3.01& 0& 22& 0& 8&  0& 1& 0& 8&  0&5 \\
~& ($\bullet$)& Polar Biol.& 2.01&0& 23&  0& 7&  0& 5&0& 4& 0&7 \\
~& ~& Sci. Total Environ.& 3.26& 0& 18& 0& 3&  0& 2&  0& 10&0&3 \\
\textbf{~}& \textbf{~}& All Journals& ~&  61& 365&18& 102& 15& 72& 18& 93&10&98 \\
Orchids& ($\bullet$)& Am. J. Bot.& 2.59& 0& 33& 0& 4& 0& 5& 0& 12& 0&12 \\
~& ($\bullet$)& Ann. Bot.-London& 3.45&1& 44&  1& 22&  0& 4&  0& 10&  0&8 \\
~& ($\bullet$)& Aust. J. Bot.& 1.20& 0& 17& 0& 9& 0& 4& 0& 2&  0&2 \\
~& ($\bullet$)& Bot. J. Linn. Soc.& 2.59& 4& 40& 0& 9&1& 11&  2& 6&1&14 \\
~& $\bullet$& Nord. J. Bot.& 0.60&  16& 12&  3& 5&4& 2&  4& 2& 5&3 \\
~& ($\bullet$)& Phytotaxa & 1.30&  16& 16&  1& 0&  1& 0& 4& 3& 10&13 \\
~& ($\bullet$)& Plant Biology& 2.32&  0& 14& 0& 4&  0& 3&  0& 6&  0&1 \\
~& ~& Plant Cell Tiss. Org.& 3.63&  0& 23& 0& 7& 0& 6&  0& 8& 0&2 \\
~& $\bullet$& Plant Syst. Evol.& 1.31&  5& 32&  1& 7&  1& 5& 2& 11&  1&9 \\
~& ~& Sci. Hortic.-Amsterdam& 1.40&  0& 42&  0& 10& 0& 8& 0& 12& 0&12 \\
& & All Journals& & 42& 273&  6& 77& 7& 48&  12& 72&  17&76 \\
Ciliates& $\bullet$& Acta Protozool.& 0.98& 16& 24& 3& 6& 6& 8& 6& 6& 1&4 \\
~& ($\bullet$)& Appl. Environ. Microb.& 3.68& 0& 22&  0& 6&  0& 6& 0& 6& 0&4 \\
~& ($\bullet$)& Aquat. Microb. Ecol.& 2.04& 0& 37& 0& 8& 0& 8& 0& 14& 0&7 \\
~& $\bullet$& Eur. J. Protistol.& 1.51&  29& 44&  8& 13& 8& 11& 7& 10& 6&10 \\
~& ($\bullet$)& Hydrobiologia & 1.99& 0& 17&  0& 5& 0& 4&  0& 5& 0&3 \\
~& $\bullet$& J. Eukaryot. Microbiol.& 2.16& 31& 45&  3& 12&  11& 10&  13& 7&  4&16 \\
~& ($\bullet$)& J. Plankton Res.& 2.44&  0& 27& 0& 5&  0& 7& 0& 10&  0&5 \\
~& ~& Mar. Ecol. Prog. Ser.& 2.55&  0& 24&  0& 7&  0& 6&  0& 4& 0&7 \\
~& ($\bullet$)& PLoS One & 3.73& 0& 25&  0& 2&  0& 1&  0& 8&  0&14 \\
~& ($\bullet$)& Protist & 4.14&  1& 28& 0& 6&  0& 6& 0& 7&  1&9 \\
\textbf{~}& \textbf{~}& All Journals& ~&  77& 293&  14& 70& 25& 67& 26& 77& 12&79 \\
Ants& ~& Anim. Behav.& 3.07&  0& 53&  0& 12& 0& 14&  0& 12&  0&15 \\
~& ~& Ecol. Entomol.& 1.95&  0& 53&  0& 11&  0& 18&  0& 13&  0&11 \\
~& ~& Environ. Entomol.& 1.31&  0& 49&  0& 13&  0& 15&  0& 14&  0&7 \\
~& ($\bullet$)& Insect. Soc.& 1.33&  0& 120& 0& 24& 0& 26&  0& 41& 0&29 \\
~& ($\bullet$)& J. Insect Sci.& 0.88&  0& 51&  0& 9& 0& 20& 0& 11& 0&11 \\
~& $\bullet$& Myrmecol. News & 2.16&  10& 71&  1& 18& 2& 10&  3& 22&  4&21 \\
~& ~& PLoS One & 3.73&  4& 120&  0& 11&  0& 23& 1& 30& 3&56 \\
~& ($\bullet$)& P. Roy. Soc. B-Biol.Sci.& 5.68& 1& 51& 0& 16& 0& 12&  1& 13& 0&10 \\
~& $\bullet$& Sociobiology & 0.58& 21& 199& 2& 52& 3& 48& 9& 52& 7&47 \\
~& $\bullet$& Zootaxa & 0.97& 36& 34& 15& 9& 4& 8&  6& 9& 11&8 \\
\textbf{~}& \textbf{~}& All Journals& ~& 72& 801& 18& 175& 9& 194&  20& 217& 25&215 \\
Snakes& ($\bullet$)& Amphibia Reptilia & 0.68& 2& 33& 2& 8& 0& 10& 0& 7& 0&8 \\
~& $\bullet$& Copeia & 0.64& 5& 31&  1& 8&  2& 8&  1& 6& 1&9 \\
~& $\bullet$& Herpetologica & 1.08&  9& 26&  3& 4&  3& 7& 2& 8& 1&7 \\
~& ~& Herpetol. Conserv. Bio.& 0.67& 0& 40& 0& 12& 0& 11& 0& 13& 0&4 \\
~& ~& J. Exp. Biol.& 3.24&  0& 25& 0& 4& 0& 10& 0& 3&  0&8 \\
~& ($\bullet$)& J. Herpetol.& 0.89& 2& 60&  0& 16&  0& 8& 2& 14& 0&22 \\
~& ~& J. Venom. Anim. Toxins& 0.55& 0& 65& 0& 16& 0& 17&  0& 15& 0&17 \\
~& ~& PLoS One & 3.73& 0& 41& 0& 2& 0& 5& 0& 16& 0&18 \\
~& ~& Toxicon & 2.92& 0& 213& 0& 52& 0& 67& 0& 47& 0&47 \\
~& $\bullet$& Zootaxa& 0.97& 36& 25& 10& 7& 4& 4& 10& 6& 12&8 \\
\textbf{~}& \textbf{~}& All Journals& ~& 54& 559& 16& 129&  9& 147&  15& 135&  14&148 \\
&&&&&&&&&&&&&\\
All Taxa& ~& All Journals&  ~& 306& 2291& 72& 553&  65& 528&  91& 594&  78&616\\ \botrule
\end{tabular*}}
{Notes: For each of the five taxa, the 10 journals included in WoS were selected that published
the largest numbers of articles on that taxon; see Online Appendix 1 for the protocols used and
Online Appendix 2 for the data available as Supplementary Material on Dryad at
\href{http://dx.doi.org/10.5061/dryad.3t761}{http://dx.doi.org/10.5061/dryad.3t761},
respectively.\newline IF $=$ impact factor; tax/nontax $=$ number of taxonomic/nontaxonomic
publications; ($\bullet$)/$\bullet$ $=$ journals publishing both taxonomic and nontaxonomic
publications in principle/on the focal taxa on a yearly basis 2009--2012.}
\end{table*}


\begin{figure*}[!p]%F1
\centerline{{\vbox to 560pt{\vfill\hbox to
412pt{\hfill\mbox{\fontsize{24}{24}\selectfont
FPO}\hfill}\vfill}}}\vspace{3pt} \caption{The average numbers of
citations received (as of 15 August 2014) by taxonomic versus
nontaxonomic publications on mosses, orchids, ciliates, ants, and
snakes based on WoS, for (a) all journals given in Table 1 and (b)
the journals in Table 1 that published both taxonomic and
nontaxonomic contributions on a yearly basis (2009--2012). Error
bars represent 1 SD. $P$-values above bars are the results of
analyses of variance comparing the numbers of citations for the
factors Taxonomy (taxonomic vs. nontaxonomic publications,
$P_{T})$, Journal ($P_{J})$, and Year ($P_{Y}); \times$
interactions among factors; $P< 0.10$ shown; $^*$ values
significant at $\alpha =0.05$. See Online Appendix 1 for the
protocols used and Online Appendix 2 for the data available as
Supplementary Material on Dryad at
\href{http://dx.doi.org/10.5061/dryad.3t761}{http://dx.doi.org/10.5061/dryad.3t761},
respectively.}\label{f1}\end{figure*}

When all taxa and all journals were included in the analyses, the average numbers of citations
were, as expected, lower for taxonomic papers than for nontaxonomic papers. This difference was
significant (Fig.~\ref{f1}a) in analysis of variance using as factors Taxonomy, Journal, and Year,
and their two- and three-level interactions calculated via Type III sum of squares; as for all
other statistical analyses, SAS 9.4 was used. In taxon-by-taxon analyses of all journals, however,
four of the five taxa were without significant differences. For the fifth taxon, ciliates,
taxonomic papers were significantly more cited than nontaxonomic ones.




Just 14 of the 47 journals published both taxonomic and nontaxonomic papers on the focal taxa on a
yearly basis in the years 2009--2012 (Table~\ref{t1}). The analyzed taxonomic publications in
these 14 journals might have experienced lower visibility than publications in the other 33
journals. This is due to the fact that the average IF 2012 of the 14 journals with both taxonomic
and nontaxonomic publications was significantly lower ($1.16\pm 0.51$ standard deviation [SD])
than the average IF of the other 33~journals ($2.66\pm 1.60$; Student's $t$-test, $P< 0.001$).

We\enlargethispage{6pt} thus repeated the analysis of variance including interactions, now
focusing on the 14 journals which published both taxonomic and nontaxonomic papers. We found that,
for these journals, taxonomic papers received more citations than nontaxonomic ones; the effect
was not significant, though ($P=0.066$; Fig.~\ref{f1}b), significance potentially being masked by
significant interactions involving Taxonomy. In the taxon-by-taxon analyses of these 14 journals,
taxonomic publications on ciliates received significantly more citations than non-taxonomic ones,
there were no significant differences for papers on mosses, ants, and snakes, whereas taxonomic
publications on orchids received significantly fewer citations than nontaxonomic ones. Because of
the correction for journal visibility, we consider the results for the 14 journals to be more
representative of the citation performance of taxonomic versus nontaxonomic \textit{per se} than
the results for all journals.

We infer that taxon-specific effects exist, at a frequency still to be determined. Citation
behavior is nontrivial to predict, and variation of citation traditions is known to occur even
across the areas of a subfield (\citealt{11Bornmann2008}; \citealt{22Erikson2014}). A frequently
mentioned effect is that the citation performance of a field depends on the size of the field
(\citealt{11Bornmann2008}; \citealt{16Casadevall2014}). For taxonomy, this has been postulated to
apply to the number of taxonomists working on a particular taxon (\citealt{32Krell2002};
\citealt{34McDade2011}). We estimated the 2009--2012 research community sizes for the five taxa
analyzed here (see Online Appendix 3 for the protocol used available as Supplementary Material on
Dryad at \href{http://dx.doi.org/10.5061/dryad.3t761}{http://dx.doi.org/10.5061/dryad.3t761}) and
found differences by up to a factor of three (mosses: 2352 authors, orchids: 1993, ciliates: 1281,
ants: 3923, and snakes: 2540). Contrary to expectations, we found no significant increase of the
number of citations with increasing community size across the five taxa\break ($P=0.166$;
covariance analysis using the means of citations, Taxonomy as factor, and community size as
covariable). Other established factors influencing citation practices include journal-dependent
ones such as the accessibility and prestige of the journal, article-dependent ones such as the
length and the number of authors on a paper, and author/reader-dependent ones such as the number
of colleagues an author is personally acquainted with (\citealt{11Bornmann2008}). Any or all of
these factors as well as interactions with the factor Taxonomy as analyzed here might have
influenced citation performances across taxa, but it is beyond the scope of our study to explore
these. Analyses of these factors for a broad array of taxa will be needed to address these issues
properly.

Importantly, we infer that publishing taxonomic contributions does not generally harm and might
even enhance the citation performance of journals, in stark contrast to previous belief (see
above). Broad monitoring will be needed to confirm this finding for taxa beyond those addressed
here.

\section*{E{\sc DITORS} C{\sc AN} I{\sc NCREASE THE} V{\sc ISIBILITY OF}
T{\sc AXONOMIC} P{\sc UBLICATIONS}}

For strengthening the impact and prospects of taxonomy, equal opportunity is needed for
taxonomists and nontaxonomists. In practice, this means that taxonomists should be able to publish
in highly visible journals (those included in WoS and with a good standing). Editors of highly
visible periodicals that include taxonomy will contribute actively to reducing the taxonomic
impediment and, considering our analyses, might on top of this do the best for their journals. Of
course, taxonomy might not fall within the scope of all journals, but among the 33 journals in
Table 1 that did not publish taxonomy on the focal taxa on a yearly basis from 2009 to 2012, 19
accept such contributions in principle and have indeed been publishing taxonomy but at a
comparatively low rate. The IF 2012 of these 19 journals that (in principle) publish taxonomy
($2.61\pm 1.64$) does, on average, not differ significantly from that of the 14 journals that do
not publish taxonomy at all ($2.73\pm 1.61$; Student's $t$-test, $P=0.84$) meaning that equal
visibility for taxonomists and nontaxonomists might, in fact, not be out of reach. In essence, for
many editors of highly visible periodicals, it might not so much be a question of changing the
scope of their journals but of increasing the frequency of taxonomic publications and thus simply
of communicating the readiness to publish taxonomy to the community. Many journals are now
embracing social media to better connect with their community, hence it is not so outlandish to
imagine that editors might soon start tweeting that they are happy to publish taxonomic papers.\vs{-14}

\section*{T{\sc AXONOMY'S} P{\sc RODUCTIVITY} W{\sc OULD BE} S{\sc UFFICIENT
TO} I{\sc NCREASE THE} N{\sc UMBER OF} P{\sc APERS IN} H{\sc IGHLY} V{\sc ISIBLE} J{\sc OURNALS}}

It is not enough, however, for editors of highly visible journals to actively invite taxonomic
contributions. A crucial question about whether increasing taxonomy's visibility will work is the
capacity of taxonomy to follow the invitation. One way to approach this issue is looking at the
growth rate of taxonomy. To have enough data points for a regression analysis, we analyzed the
period 1993--2012. Over this period, the number of taxonomic publications in journals included in
WoS grew steadily, and the growth is better explained by an exponential than by a linear model,
for all organisms (Fig.~\ref{f2}a; see Online Appendices 4 and 5 for the protocols of database
queries and statistical analysis, available as Supplementary Material on Dryad at
\href{http://dx.doi.org/10.5061/dryad.3t761}{http://dx.doi.org/10.5061/dryad.3t761}, respectively)
as well as for plants, microorganisms, and animals (Fig.~\ref{f2}b). Possibly even more
importantly, taxonomy as represented in WoS grew over the same period with greater speed than
biology, again for all organisms as well as for plants, microorganisms, and animals
(Fig.~\ref{f2}d) despite the decelerated growth rate of all biodiversity research in the past few
years (\citealt{46Stork2014}). This greater speed in growth makes it plausible that editors
publishing taxonomy might indeed boost their journals.

\begin{figure*}[!t]%F2
\centerline{{\vbox to 572pt{\vfill\hbox to
412pt{\hfill\mbox{\fontsize{24}{24}\selectfont
FPO}\hfill}\vfill}}}\vspace{3pt} \caption{The number of taxonomic
publications (1993--2012) included in (a, b) WoS, and (c) ZR, on
(a) all organisms, (b) plants, microorganisms, and animals, and
(c) just animals; in addition, in (b, c) the numbers are shown for
animals when excluding the journal Zootaxa. The results of
regression analyses comparing the $R^{2}$ of linear (lin) and
exponential (exp) functions are added. d) The portion of taxonomic
publications on all organisms and on plants, microorganisms, and
animals included in WoS of all biological publications in WoS. See
Online Appendix 4a--h for the database query protocols used and
Appendix 5 for the regression analysis results available as
Supplementary Material on Dryad at
\href{http://dx.doi.org/10.5061/dryad.3t761}{http://dx.doi.org/10.5061/dryad.3t761},
respectively. Years are given as relative years as used in the
regression analyses: $2=1993$, $21=2012$.} \label{f2}
\end{figure*}

Another approach to the question of taxonomy's capacity is whether there are sufficient
publications in total, that is, including in journals not indexed in WoS. This might be especially
relevant to editors of WoS-indexed journals who decide to publish taxonomy on a frequent basis. To
our knowledge, a comprehensive taxonomic literature database is available just for animals,
Zoological Record (ZR). For 2012, the latest year considered here, ZR lists 2.1 times more
publications on animal taxonomy than WoS (Fig.~\ref{f2}b, c). This indicates that already in the
short term, there is sufficient taxonomic publication output for editors of highly visible
journals to indeed increase their share in taxonomy. Also, just as in WoS (Fig.~\ref{f2}b), animal
taxonomy grew in ZR (in line with \citealt{49Tancoigne2013}) with an exponential rather than
linear growth rate~(Fig.~\ref{f2}c).\vs{-3}


\section*{T{\sc AXONOMY} C{\sc AN} A{\sc LSO BE} P{\sc UBLISHED BY}
N{\sc ONTAXONOMISTS}}


There is, however, realistic hope that the potential to publish taxonomic papers might be even
greater than the ZR analysis suggests. First, the dichotomy between taxonomists and nontaxonomists
does not always exist. Some scientists do both sorts of research, either because of diverse
interests in the first place or because of publishing on ecology, evolution, or biogeography as a
survival strategy of taxonomists in today's IF-ruled scientific system (\citealt{40Samyn2012};
also see Halme et al. 2015). Quantifying their number appears difficult, but these biologists
could increase easily their activity in taxonomy, once aware that publishing taxonomy could be
beneficial to their career.

Second, there are biologists who are spending a considerable amount of time and money on topics
pertinent to species delimitation but have never included taxonomic acts in their publications,
which focus on phylogeny or phylogeography. This has been criticized as a diversion of funds
(\citealt{58Wheeler2004}). However, it also means there is a ready capital from which taxonomy
could start profiting: Of 353 sets of specimens included in a literature analysis of studies that
reported arthropod diversity at the species level, the need for taxonomic change was revealed for
123, but taxonomic change was published for just 48 (\citealt{42Schlick-Steiner2010}). Given that
interdisciplinarity between, for example, physics and biology is now common
(\citealt{56West2014}), it should be relatively easy to achieve this cross-talk between
taxonomists and nontaxonomist biologists.

On the whole, the capacity for increased publication of taxonomy in highly visible journals seems
to be there. Accepting that the potential exists, there is still a question of whether taxonomy's
flexibility will be sufficient for a change in publication culture to be realized.\vs{-3}

\section*{T{\sc HE} C{\sc OMMUNITY} W{\sc OULD} L{\sc IKELY} U{\sc SE THE} C{\sc HANCE
TO} I{\sc NCREASE} T{\sc AXONOMY'S} V{\sc ISIBILITY}}

It is difficult to predict to which degree all those publishing taxonomy would accept changes in
editorial policy, but there is a prominent example that speaks for optimism. The journal Zootaxa
was founded in 2001 to ``help taxonomists overcome the taxonomic impediment by enabling them to
describe biodiversity in a rapid and efficient way'' (\citealt{59Zhang2011}). Zootaxa was included
in WoS in 2004 with a subsequent increase in the numbers of contributions published and citations
received (\citealt{59Zhang2011}). Zootaxa has published a considerable fraction of the overall
number of taxonomic papers on animals in WoS and in ZR (Fig.~\ref{f2}b, c;
\citealt{49Tancoigne2013}). This suggests that taxonomists indeed would use also other chances of
publishing in highly visible journals, should the opportunity arise. The resulting shift from
aiming at low visibility to targeting highly visible journals will be very important for
taxonomists in working toward both an improved image (\citealt{15Carbayo2011}) and an improved
measure of their scientific impact (\citealt{3Agnarsson2007}).

\section*{C{\sc ONCLUSIONS}}

Criticisms of the use of bibliometric tools such as the IF in decisions about who gets funded and
who gets academic jobs are justified (\citealt{6Benitez2014}; \citealt{22Erikson2014};
\citealt{39Pyke2014}). However, these tools are currently used widely and, as long as this is the
case, taxonomy would benefit from a positive bibliometric performance. We suggest that changes in
publication culture might help reduce the taxonomic impediment. Editors of highly visible journals
in biology could help (i) increase the visibility of taxonomic publications by encouraging
taxonomists to publish in their journals (thereby generally not harming but possibly boosting
their journals), and (ii) increase total taxonomic output by making it attractive for scientists
working in species delimitation (with their primary focus different from taxonomy) to publish the
taxonomic consequences of their research.

The task of taxonomic authors, in turn, will be to follow the invitation and to submit indeed
their best papers to the best-visible journals available for submission---just as authors of
nontaxonomic papers do. These actions together would very likely increase the citation strength of
taxonomy as measured by the IF and similar tools and thus improve taxonomists' chances in
competing for academic positions and research funding.

Here, we have revisited one seemingly well-established explanation for the taxonomic impediment,
taxonomy's poor citation performance, with surprising results. We personally doubt that other
explanations for the taxonomic impediment such as difficult job and funding situations would
likewise turn out to be preconceived ideas---but evidence-based scrutiny is needed.


\section*{S{\sc UPPLEMENTARY} M{\sc ATERIAL}}

Data available from the Dryad Data Repository:
\href{http://dx.doi.org/10.5061/dryad.3t761}{http://dx.doi.org/10.5061/dryad.3t761}.

\section*{F{\sc UNDING}}

This work was supported by the Austrian Science Fund, FWF (grant numbers P23409 to B.C.S.-S,
P25955 to F.M.S.); and the science fund of the Autonomous Province of South Tyrol (grant number
40.3/22306/27.01.2014 to W.A.).

\section*{A{\sc CKNOWLEDGMENTS}}

Many thanks to Clemens Folterbauer and Magdalena Tratter for assistance with data preparation in a
pilot study, to Ottmar Holdenrieder for bibliometric training, to Thomas Dejaco for help with
figure preparation, to Brian Golding for valuable discussion, and to Frank Anderson, Benoit
Dayrat, Sarah Samadi, and one anonymous reviewer for their helpful comments.\vs{-8}

\begin{thebibliography}{}

\bibitem[Adler(2009)Adler ]{1Adler2009}
Adler K.B. 2009. Impact factor and its role in academic promotion. Am. J. Respir. Cell Mol. Biol.
41:127--127.

\bibitem[Adler and Harzing(2009)Adler Harzing ]{2Adler2009}
Adler N.J., Harzing A.W. 2009. When knowledge wins: transcending the sense and nonsense of
academic rankings. Acad. Manag. Learn. Educ. 8:72--95.

\bibitem[Agnarsson and Kuntner(2007)Agnarsson Kuntner ]{3Agnarsson2007}
Agnarsson I., Kuntner M. 2007. Taxonomy in a changing world: seeking solutions for a science in
crisis. Syst. Biol. 56:531--539.

\bibitem[(2009)Anderson Majka ]{4Anderson2009}
Anderson R.S., Majka C.G. 2009. Biodiversity and biosystematic research in a brave new 21st
century information-technology world. ZooKeys 22:1--4.

\bibitem[(2014)Bebber Wood Barker Scotland ]{5Bebber2014}
Bebber D.P., Wood J.R.I., Barker C., Scotland R.W. 2014. Author inflation masks global capacity
for species discovery in flowering plants. New Phytol. 201:700--706.

\bibitem[Ben\'{\i}tez(2014)Ben\'{\i}tez ]{6Benitez2014}
Ben\'{\i}tez H.A. 2014. Is the fever for high impact a disadvantage for systematists? Neotrop.
Entomol. 43:295--297.

\bibitem[Bilton(2014)Bilton ]{7Bilton2014}
Bilton D.T. 2014. What's in a name? What have taxonomy and systematics ever done for us? J. Biol.
Educ. 48:116--118.

\bibitem[Bininda-Emonds(2011)Bininda-Emonds ]{8Bininda-Emonds2011}
Bininda-Emonds O.R.P. 2011. Supporting species in ODE: explaining and citing. Org. Divers. Evol.
11:1--2.

\bibitem[Boero(2001)Boero ]{9Boero2001}
Boero F. 2001. Light after dark: the partnership for enhancing expertise in taxonomy. Trends Ecol.
Evol. 16:266.

\bibitem[Boero(2010)Boero ]{10Boero2010}
Boero F. 2010. The study of species in the era of biodiversity: a tale of stupidity. Diversity
2:115--126.

\bibitem[Bornmann and Daniel(2008)Bornmann Daniel ]{11Bornmann2008}
Bornmann L., Daniel H.D. 2008. What do citation counts measure? A review of studies on citing
behavior. J. Doc. 64:45--80.

\bibitem[Bortolus(2008)Bortolus ]{12Bortolus2008}
Bortolus A. 2008. Error cascades in the biological sciences: the unwanted consequences of using
bad taxonomy in ecology. Ambio 37:114--118.

\bibitem[Boykin et al.(2011)Boykin Armstrong Kubatko Barro ]{13Boykin2011}
Boykin L.M., Armstrong K.F., Kubatko L., Barro P.D. 2011. Species delimitation and global
biosecurity. Evol. Bioinform. 8:1--37.

\bibitem[Brumback(2012)Brumback ]{14Brumback2012}
Brumback R.A. 2012. ``3 . . 2 . . 1 . . impact [factor]: target [academic career] destroyed!'':
just another statistical casualty. J. Child Neurol. 27:1565--1576.

\bibitem[Carbayo and Marques(2011)Carbayo Marques ]{15Carbayo2011}
Carbayo F., Marques A.C. 2011. The costs of describing the entire animal kingdom. Trends Ecol.
Evol. 26:154--155.

\bibitem[Casadevall and Fang(2014)Casadevall Fang ]{16Casadevall2014}
Casadevall A., Fang F.C. 2014. Causes for the persistence of impact factor mania. mBio 5:5.

\bibitem[(2013)Costello Wilson Houlding ]{17Costello2013}
Costello M.J., Wilson S., Houlding B. 2013. More taxonomists describing significantly fewer
species per unit effort may indicate that most species have been discovered. Syst. Biol.
62:616--624.

\bibitem[De Carvalho et~al.(2007)]{18De2007}
De Carvalho M.R., Bockmann F.A., Amorim D.S., Brandao C.R.F., Vivo M.d., Figueiredo J.L.d.,
Britski H.A., Pinna M.C.C.d., Menezes N.A., Marques F.P.L., Papavero N., Cancello E.M., Crisci
J.V., McEachran J.D., Schelly R.C., Lundberg J.G., Gill A.C., Britz R., Wheeler Q.D., Stiassny
M.L.J., Parenti L.R., Page L.M., Wheeler W.C., Faivovich J., Vari R.P., Grande L., Humphries C.J.,
DeSalle R., Ebach M.C., Nelson G.J. 2007. Taxonomic impediment or impediment to taxonomy?
A~commentary on systematics and the cybertaxonomic-automation paradigm. Evol. Biol. 34:140--143.

\bibitem[De Carvalho et~al.(2014)]{19De2014}
De Carvalho M.R., Ebach M.C., Williams D.M., Nihei S.S., Rodrigues M.T., Grant T., Silveira L.F.,
Zaher H., Gill A.C., Schelly R.C., Sparks J.S., Bockmann F.A., Seret B., Ho H.C., Grande L.,
Rieppel O., Dubois A., Ohler A., Faivovich J., Assis L.C.S., Wheeler Q.D., Goldstein P.Z., de
Almeida E.A.B., Valdecasas A.G., Nelson G. 2014. Does counting species count as taxonomy? On
misrepresenting systematics, yet again. Cladistics 30:322--329.

\bibitem[Ebach et al.(2011)Ebach Valdecasas Wheeler ]{20Ebach2011}
Ebach M.C., Valdecasas A.G., Wheeler Q.D. 2011. Impediments to taxonomy and users of taxonomy:
accessibility and impact evaluation. Cladistics 27:550--557.

\bibitem[(2013)Elias ]{21Elias2013}
Elias S.A. 2013. A brief history of the changing occupations and demographics of coleopterists
from the 18th through the 20th century. J. Hist. Biol. 47:213--242.

\bibitem[Erikson and Erlandson(2014)Erikson Erlandson ]{22Erikson2014}
Erikson M.G., Erlandson P. 2014. A taxonomy of motives to cite. Soc. Stud. Sci. 44:625--637.

\bibitem[Eyre-Walker and Stoletzki(2013)Eyre-Walker Stoletzki ]{23Eyre-Walker2013}
Eyre-Walker A., Stoletzki N. 2013. The assessment of science: the relative merits of
post-publication review, the impact factor, and the number of citations. PLoS Biol. 11:e1001675.

\bibitem[Foley(2013)Foley ]{24Foley2013}
Foley J.A. 2013. Peer review, citation ratings and other fetishes. Springer Sci. Rev. 1:1--3.

\bibitem[Gaston and May(1992)Gaston May ]{25Gaston1992}
Gaston K.J., May R.M. 1992. Taxonomy of taxonomists. Nature 356:281--282.

\bibitem[(2002)Godfray ]{26Godfray2002}
Godfray H.C.J. 2002. Challenges for taxonomy. Nature 417:17--19.

\bibitem[(2015)Halme Kuusela Jusl\'{e}n ]{27Halme2015}
Halme P., Kuusela S., Jusl\'{e}n A. 2015. Why taxonomists and ecologists are not, but should be,
carpooling? Biodivers. Conserv. 1--6.

\bibitem[(2011)Haszprunar ]{28Haszprunar2011}
Haszprunar G. 2011. Species delimitations -- not `only descriptive'. Org. Divers. Evol.
11:249--252.

\bibitem[(2002)Hopkins Freckleton ]{29Hopkins2002}
Hopkins G.W., Freckleton R.P. 2002. Declines in the numbers of amateur and professional
taxonomists: implications for conservation. Anim. Conserv. 5:245--249.

\bibitem[Kaushal and Jeschke(2013)Kaushal Jeschke ]{30Kaushal2013}
Kaushal S.S., Jeschke J.M. 2013. Collegiality versus competition: How metrics shape scientific
communities. BioScience 63:155--156.

\bibitem[Krell(2000)Krell ]{31Krell2000}
Krell F.-T. 2000. Impact factors aren't relevant to taxonomy. Nature 405:507--508.

\bibitem[Krell(2002)Krell ]{32Krell2002}
Krell F.-T. 2002. Why impact factors don't work for taxonomy. Nature 415:957--957.

\bibitem[Lester et al.(2014)Lester Brown Edwards Holwell Pawson Ward Watts ]{33Lester2014}
Lester P.J., Brown S.D.J., Edwards E.D., Holwell G.I., Pawson S.M., Ward D.F., Watts C.H. 2014.
Critical issues facing New Zealand entomology. N. Z. Entomol. 37:1--13.

\bibitem[McDade et al.(2011)McDade Maddison Guralnick Piwowar Jameson Helgen Herendeen Hill Vis ]{34McDade2011}
McDade L.A., Maddison D.R., Guralnick R., Piwowar H.A., Jameson M.L., Helgen K.M., Herendeen P.S.,
Hill A., Vis M.L. 2011. Biology needs a modern assessment system for professional productivity.
BioScience 61:619--625.

\bibitem[Mora et al.(2011)Mora Tittensor Adl Simpson Worm ]{35Mora2011}
Mora C., Tittensor D.P., Adl S., Simpson A.G.B., Worm B. 2011. How many species are there on earth
and in the ocean? PLoS Biol. 9:e1001127.

\bibitem[(1993)Parnell ]{36Parnell1993}
Parnell J. 1993. Plant taxonomic research, with special reference to the tropics: problems and
potential solutions. Conserv. Biol. 7:809--814.

\bibitem[Patterson(2009)Patterson ]{37Patterson2009}
Patterson M. 2009. Is the end in cite? EMBO Rep. 10:1186--1186.

\bibitem[(2011)Pearson Hamilton Erwin ]{38Pearson2011}
Pearson D.L., Hamilton A.L., Erwin T.L. 2011. Recovery plan for the endangered taxonomy
profession. BioScience 61:58--63.

\bibitem[Pyke(2014)Pyke ]{39Pyke2014}
Pyke G.H. 2014. Evaluating the quality of taxonomic publications: a simple alternative to
citations and effort. BioScience.

\bibitem[Samyn and Clock(2012)Samyn Clock ]{40Samyn2012}
Samyn Y., Clock O.D. 2012. No name, no game. Eur. J. Taxon. 10:1--3.

\bibitem[Schekman(2013)Schekman ]{41Schekman2013}
Schekman R.P.M. 2013. Reforming research assessment. eLife 2:e00855.

\bibitem[Schlick-Steiner et al.(2010)Schlick-Steiner Steiner Seifert Stauffer Christian Crozier ]{42Schlick-Steiner2010}
Schlick-Steiner B.C., Steiner F.M., Seifert B., Stauffer C., Christian E., Crozier R.H. 2010.
Integrative taxonomy: a multisource approach to exploring biodiversity. Annu. Rev. Entomol.
55:421--438.

\bibitem[Shashank and Meshram(2014)Shashank Meshram ]{43Shashank2014}
Shashank P.R., Meshram N.M. 2014. Impact factor-driven taxonomy: deterrent to Indian taxonomists?
Curr. Sci. 106:10--10.

\bibitem[Simons(2008)Simons ]{44Simons2008}
Simons K. 2008. The misused impact factor. Science 322:165.

\bibitem[(2013)Sluys ]{45Sluys2013}
Sluys R. 2013. The unappreciated, fundamentally analytical nature of taxonomy and the implications
for the inventory of biodiversity. Biodivers. Conserv. 22:1095--1105.

\bibitem[Stork and Astrin(2014)Stork Astrin ]{46Stork2014}
Stork H., Astrin J.J. 2014. Trends in biodiversity research -- a bibliometric assessment. Open J.
Ecol. 4:354--370.

\bibitem[Sundberg and Strand(2009)Sundberg Strand ]{47Sundberg2009}
Sundberg P., Strand M. 2009. Taxonomic inflation or taxonomist deflation? A comment on Dubois.
Biol. J. Linn. Soc. 96:712--714.

\bibitem[Swiss Academy of Sciences(2007)Swiss Academy of Sciences ]{48Swiss2007}
Swiss Academy of Sciences 2007. The future of systematics in Switzerland: systematics as a key
discipline in biology. J. Zool. Syst. Evol. Res. 45:285--288.

\bibitem[Tancoigne and Dubois(2013)Tancoigne Dubois ]{49Tancoigne2013}
Tancoigne E., Dubois A. 2013. Taxonomy: no decline, but inertia. Cladistics 29:567--570.

\bibitem[Valdecasas(2011)Valdecasas ]{50Valdecasas2011}
Valdecasas A.G. 2011. An index to evaluate the quality of taxonomic publications. Zootaxa
2925:57--62.

\bibitem[Valdecasas et al.(2000)Valdecasas Castroviejo Marcus ]{51Valdecasas2000}
Valdecasas A.G., Castroviejo S., Marcus L.F. 2000. Reliance on the~citation~index undermines the
study of biodiversity. Nature 403:698.

\bibitem[Vale(2012)Vale ]{52Vale2012}
Vale R.D. 2012. Evaluating how we evaluate. Mol. Biol. Cell\break 23:3285--3289.

\bibitem[Venu and Sanjappa(2011)Venu Sanjappa ]{53Venu2011}
Venu P., Sanjappa M. 2011. The impact factor and taxonomy. Curr. Sci. 101:1397--1397.

\bibitem[W\"{a}gele et al.(2011)W\"{a}gele Klussmann-Kolb Kuhlmann Haszprunar Lindberg Koch W\"{a}gele ]{54Wagele2011}
W\"{a}gele H., Klussmann-Kolb A., Kuhlmann M., Haszprunar G., Lindberg D., Koch A., W\"{a}gele
J.W. 2011. The taxonomist -- an\vvp{} endangered race. A practical proposal for its survival. Front.
Zool. 8:25.

\bibitem[Werner(2006)Werner ]{55Werner2006}
Werner Y.L. 2006. The case of impact factor versus taxonomy: a proposal. J. Nat. Hist.
40:1285--1286.

\bibitem[West(2014)West ]{56West2014}
West G.B. 2014. A theoretical physicist's journey into biology: from quarks and strings to cells
and whales. Phys. Biol. 11:053013.

\bibitem[(2014)Wheeler ]{57Wheeler2014}
Wheeler Q. 2014. Are reports of the death of taxonomy an exaggeration? New Phytol. 201:370--371.

\bibitem[Wheeler(2004)Wheeler ]{58Wheeler2004}
Wheeler Q.D. 2004. Taxonomic triage and the poverty of phylogeny. Philos. Trans. R. Soc. Lond. B
Biol. Sci. 359:571--583.

\bibitem[Zhang(2011)Zhang ]{59Zhang2011}
Zhang Z.Q. 2011. Accelerating biodiversity descriptions and transforming taxonomic publishing: the
first decade of Zootaxa. Zootaxa 2896:1--7.

\end{thebibliography}

\end{document}