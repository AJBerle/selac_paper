\documentclass{article}

\usepackage{fullpage}
\usepackage[doublespacing]{setspace}
\usepackage{amsmath, amssymb,amsfonts}
\usepackage[small]{caption}
\usepackage{graphicx}
\usepackage{xspace}
\usepackage{natbib}
\usepackage{datetime} %provides \currenttime command

\graphicspath{{./Figures/}}
\DeclareGraphicsExtensions{.pdf, .jpg, .png}



%%%%%%%%%%%%%%%%%%%%%%%%%%%%%%%%%Local Commands %%%%%%%%%%%%%%%%%%%%%%%%%%%%%
\newcommand{\LogN}{\ensuremath{\text{LogN}}\xspace}

\newcommand{\EE}{\mathbb{E}} %use for expectation function E()
\newcommand{\setG}{\ensuremath{\mathbb{G}}\xspace}
\newcommand{\setP}{\ensuremath{\mathbb{P}}\xspace}
\newcommand{\qij}{\ensuremath{q_{i,j}}\xspace}
\newcommand{\qji}{\ensuremath{q_{i,j}}\xspace}
\newcommand{\pij}{\ensuremath{p_{i,j}}\xspace}
\newcommand{\muij}{\ensuremath{\mu_{i,j}}\xspace}
\newcommand{\fij}{\ensuremath{f_{i,j}}\xspace}
\newcommand{\Pii}{\ensuremath{\pi_{i}}\xspace}
\newcommand{\Pij}{\ensuremath{\pi_{j}}\xspace}
\newcommand{\Pivec}{\ensuremath{\Vec{\pi}}\xspace}
\newcommand{\Pivechat}{\ensuremath{\hat{\Pivec}}\xspace}
\newcommand{\Piihat}{\ensuremath{\hat{\pi}_i}\xspace}
\newcommand{\Pijhat}{\ensuremath{\hat{\pi}_j}\xspace}
\newcommand{\Wi}{\ensuremath{{W_i}}\xspace}
\newcommand{\Wj}{\ensuremath{{W_j}}\xspace}
\newcommand{\Ne}{\ensuremath{{N_e}}\xspace}
% \renewcommand{\ng}{\ensuremath{{n_g}}\xspace}
\newcommand{\DeltaAIC}{\ensuremath{\Delta\text{AIC}}\xspace}
\newcommand{\GTR}{GTR+$\Gamma$\xspace}

\newcommand{\jmax}{\ensuremath{{j_{\max}}}\xspace}
\newcommand{\kmax}{\ensuremath{{k_{\max}}}\xspace}
\newcommand{\cvec}{\ensuremath{\Vec{c}}\xspace}
\newcommand{\cveci}{\ensuremath{\cvec_i}\xspace}
\newcommand{\cvecj}{\ensuremath{\cvec_j}\xspace}

\newcommand{\acvec}{\ensuremath{a\left(\Vec{c}\right)}\xspace}
\newcommand{\acivec}{\ensuremath{a\left(\cveci\right)}\xspace}
\newcommand{\acvecg}{\ensuremath{a\left(\vec{c}_{i,g}\right)}\xspace}
\newcommand{\aip}{\ensuremath{a_{i,p}}\xspace}

\newcommand{\avec}{\ensuremath{\Vec{a}}\xspace}
\newcommand{\aveci}{\ensuremath{\Vec{a}_i}\xspace}
\newcommand{\aivec}{\aveci}
\newcommand{\aivecg}{\ensuremath{{\avec}_{i,g}}\xspace}

\newcommand{\avecj}{\ensuremath{\Vec{a}_j}\xspace}
\newcommand{\acvecj}{\ensuremath{a\left(\cvecj\right)}\xspace}


\newcommand{\aopt}{\ensuremath{a^*}\xspace}
\newcommand{\aoptip}{\ensuremath{\aopt_{i,p}}\xspace}
\newcommand{\aoptp}{\ensuremath{\aopt_p}\xspace}
\newcommand{\aoptpg}{\ensuremath{\aopt_{p,g}}\xspace}
\newcommand{\aoptvec}{\ensuremath{\Vec{a}^*}\xspace}
\newcommand{\aoptvecg}{\ensuremath{{{\aoptvec}_g}}\xspace}
\newcommand{\aobs}{\ensuremath{a_{\text{obs}}}\xspace}
\newcommand{\aobsvec}{\ensuremath{\Vec{a}_{\text{obs}}}\xspace}
\newcommand{\aobsvecg}{\ensuremath{{\avec}_{\text{obs},g}}\xspace}



\newcommand{\aj}{\ensuremath{a__j}\xspace}
\newcommand{\ajvec}{\ensuremath{\Vec{a}_{j}}\xspace}
\newcommand{\ajvecg}{\ensuremath{{\ajvec}_{,g}}\xspace}

\newcommand{\phig}{\ensuremath{\phi_{g}}\xspace}

\newcommand{\Cost}{\ensuremath{\text{Cost}}\xspace}
\newcommand{\Costaobsvec}{\ensuremath{\text{Cost}(\aobsvec)}\xspace}
\newcommand{\Costavec}{\ensuremath{\text{Cost}(\avec)}\xspace}
\newcommand{\Costaveci}{\ensuremath{\text{Cost}(\aveci)}\xspace}
\newcommand{\Costavecj}{\ensuremath{\text{Cost}(\avecj)}\xspace}

\newcommand{\Costcveci}{\ensuremath{\text{Cost}(\cveci)}\xspace}
\newcommand{\Costcvecj}{\ensuremath{\text{Cost}(\cvecj)}\xspace}

\newcommand{\Func}{\ensuremath{\text{Benefit}}\xspace}
\newcommand{\Funcaobsvec}{\ensuremath{\Func(\aobsvec|\aoptvec)}\xspace}
\newcommand{\Funcavec}{\ensuremath{\Func(\avec|\aoptvec)}\xspace}
\newcommand{\Funcaveci}{\ensuremath{\Func(\aveci|\aoptvec)}\xspace}
\newcommand{\Funcavecj}{\ensuremath{\Func(\avecj|\aoptvec)}\xspace}
\newcommand{\Funcaoptvec}{\ensuremath{\Func(\aoptvec)}\xspace}

\newcommand{\Funccvec}{\ensuremath{\Func(\cvec|\aoptvec)}\xspace}
\newcommand{\Funccveci}{\ensuremath{\Func(\cveci|\aoptvec)}\xspace}

\renewcommand{\ng}{\ensuremath{{n_g}}\xspace}


\title{SelAC Paper}
\date{}
\begin{document}
\maketitle

\section*{Introduction}
\begin{enumerate}
\item Phylogenetic methods play an important role in many fields of biology and medicine.
\item Essentially all phylogenetic approaches use a substitution matrix $Q = \left\{\qij\right\}$  to model evolution, where
  \begin{equation*}
   \qij  = \text{Substitution rate from state $i$ to $j$.}
  \end{equation*}
\item Most models, e.g.~F84, GTR, and GY94, use a `time reversible matrix' where $\pi_i \qij = \pi_j \qji $ for all $i \neq j$.
\item Time reversible matrices  (TRM)  are very computationally efficient to employ.
\item TRM were initially derived under the assumption of neutrality but have been extended heuristically to describe non-neutral evolution.
\item However, non-neutral evolution is not a time reversible process, thus TRM models are unlikely to  accurately desribe evolutionary behavior when natural selection occurs.
To illustrate the disconnect between time reversible models and non-neutral evolution we use a simplified version of the extremely popular Goldman and Yang (1994)\cite{GoldmanAndYang94} (GY94) codon level model.
In their model,
    \begin{align*}
      q_{i,j} &%= \text{Transition rate from $i \to j$} 
         = \begin{cases}
           0 & \text{$i$ and $j$ differ by more than one substitution}\\
           \Pijhat & \text{Synonymous (S) substitution} \\
           \omega \Pijhat & \text{Non-Synonymous (NS) substitution} \\
         \end{cases}
         \intertext{Where,}
         \omega &= \text{`Selection' term applied to all NS substitutions}\\
         \Pijhat &= \text{Equilibrium frequency of codon $i$}
       \end{align*}
When $\omega <1$ the GY94 model is purported to describe evolution under `purifying' selection where S substitutions are favored over NS substitutions.
However, the model has the following behavior
    \begin{enumerate}
    \item If $i$ is the current state, GY94 implies selection favoring $i$. \label{pureone}
    \item However, if NS substitution occurs, \ref{pureone} still applies and selection now favoring new state $j$!
    \end{enumerate}
Thus, the behavior of GY94 is actually not consistent with a constant selective environment, but instead is consistent with a system where the directionality of natural selection and a NS substitution occurs simultaneously.
Similar inconsistencies occur when $\omega < 1$.
\item The counter argument to the fact that non-neutral evolution is not time reversible is the observation that TRM do a good job reconstructing phylogenetic trees. 
However, since the results of TRM models are rarely compared to non-TRM models (see [CITATIONS] for notable exceptions), how well they perform relative to more realistic models is an open question.
\item In this study we develop a non-TRM (NTRM) model where the substitution rate of an allele is based on the substitution probability of an allele in the presence of selection for reducing protein synthesis costs and genetic drift, per standard models of population genetics.
\item In developing our model, we assume that for each protein coding gene there is a single amino acid sequence which executes its intended function better than any other sequence, i.e. is optimal.
We also assume that the functionality of other amino acid sequences declines as the physiochemical properties of the sequence deviates from that of the optimal sequence.
\item We describe how functionality  declines with physiochemical distance using a Taylor series expansion and a set of weighting terms, which we estimate.
\item Because we assume that a protein's functionality is a declining function of the product of the physiochemical distances of each of the protien's amino acid from the optimal, we can treat the evolution at each amino acid position in a site independent manner. 
An approach which is almost universally used in TRM models.
\item As a result, unlike most phylogenetic approaches, our model requires 20 different 20x20 rate matricies, one for when each amino acid is the optimal one.
\item Even though our model requires a large number of matricies, because of our assumption that a protein's functionality is a declining function of physiochemical distance from the optimum, we are able to parameterize our 20 matricies using only a handful of parameters which we estimate from the data.
\item Two additional key assumption of our model is that (a) the organism has an average target production rate $\phi$ for the functionality provided by each protein and (b) that protein synthesis is under some form of  regulatory control such that the this average functionality production target is met.
As a result, the relative rate of protein synthesis increases as the sequence's functionality declines due to deviation from the optimal sequence.
This behavior, in turn, means that the energetic cost of protein synthesis for an allele deviating from the optimal sequence increases with the target production rate $\phi$.
For example, a protein encoding allele which has a 10\% reduction in functionality will have the same energetic burden relative to its optimal sequence as a protein encoding allele of similar length which has a 20\% reduction in fuctionality but whose target production rate is 1/2 of the first protein.
\item In its current formulation, our model is only applicable to protein coding sequences.
However, it should be applicable to non-coding sequences so long as one has a mapping function between gene sequence and gene function.
\end{enumerate}





\section*{Methods}
\subsection*{Allele Substitution Model}
\subsubsection*{Defining the Mutation Rate Matrix $\Mu$: }
We begin by defining a time reversible model for mutation rates among codons, \Mu, a 64x64 matrix, where entries \mu_ij describe the mutation rate from codon $i$ to $j$.
We seed these rates according to a pre-defined substitution-rate model (e.g., JC, GTR) for a four-state nucleotide model, which describe the instantaneous rate of change from nucleotide $i$ to $j$.
For simplicity we assume that the mutations occur independent between nucleotides within a codon. 
For codons that differ only by one nucleotide, the rate between codons is equal to the rate between the pair of nucleotides.
For any pair of codons that differ by more than one nucleotide, the rates are set to zero, since changes involving two or more nucleotides during time deltaT have probabilities on the order of $\delta t^2$. 

\subsubsection*{Defining Protein Synthesis Cost-Benefit Function $\eta$: }
Because our model assumes that natural selection favors genotypes that are able to meet their metabolic requirements more efficiently than their competitors, our framework centers on the cost-benefit function of a gene $g$, $\eta_g$, and the organisms average target production rate of the functionality provided by gene $g$, $\phig$.
This is because the average amount of energy an organism spends to met its target functionality for a gene $g$ is $\eta_g \times \phig$.
 
\paragraph*{Defining the Cost Function}
Generally speaking, protein synthesis involves both direct and indirect assembly costs.
Direct costs consist of the high energy phosphate bonds in ATP or GTP's used to assemble the ribosome on the mRNA, charge tRNA's for elongation, move the ribosome forward along the transcript, and terminate protein synthesis.
Indirect costs are many and consist of the cost of amino acid synthesis as well as synthesis of the protein assembly infrastructure such as ribosomes, aminoacyl-tRNA synthetases, tRNAs, mRNAs, etc.
Direct synthesis costs are the same for all proteins of the same length.
For simplicity, in this study we ignore any indirect costs of protein synthesis that vary between genotypes.
As a result, 
\begin{align}
\label{eq:defineCost}
  \Costcveci  &= \text{Energetic cost of protein synthesis.}\\
  &= C_1 + C_2 n
\end{align}
where, $C_1$ and $C_2$ represent the direct and indirect costs in ATPs of ribosome initiation and peptide elongation, respectively.
When sequence specific costs, such as ribosome pausing times, are included with sequence specific benefits, the probability of a mutant allele fixing is no longer independent of the rest of the sequence.
As a result, our site independent assumption is violated and the fitting of our model becomes much more complex.
For simplicity, in this study we only consider the direct costs of protein assembly and, thus, $C_1 = C_2 = 4 \text{ATP}$.
 
\paragraph*{Defining the Benefit Function: }
In order to link genotype to protein function, we define a benefit function which measures the functionality of the peptide encoded by \cveci, i.e. $a(\cveci) = \aveci$ relative to the the optimal sequence $\aoptvec$.
By definition, we set $\Funcaoptvec = 1$ and assume $\Funcaveci < 1$ for all $\aveci$ other than the optimal sequence.
How protein functionality declines with deviation from \aoptvec, is an overwhelmingly complex problem and likely varies between different categories of proteins.
Instead of claiming to accurately model this relationship between genotype and protein function, we will fit a Taylor Series expansion to our data in order to approximate its general behavior.
We will also assume a form that results in independent evolution between sites within a gene.
Alternative forms of \Funcaveci can, of course, be explored by other researchers.

To begin, we assume that each amino acid makes a similar contribution to protein function (an assumption that can be relaxed) and that this contribution declines as an inverse function of physiochemical distance.
More specifically, we assume 
\begin{equation}
\Funcaveci = \left(\frac{1}{\ng} \sum_p^\ng{} f\left(d\left(\aip, \aoptp\right)\right)\right)^{-1}
\end{equation}
where $\ng$ is the length of the protein, $d(\aip, \aoptp)$ is the physiochemical distance between the amino acid encoded in gene $i$ for position $p$ and $\aoptp$ is the optimal amino acid for that position of the protein, and $1/f(d)$ describes how the contribution of amino acid to protein function declines with $d$.

How $f(d)$ changes with $d$ is unknown, as a result we use a combination of a Taylor Series expansion and random effect to describe the relationship between $f$ and $d$.
Given our assumption that \Funcaoptvec = 1 and noting that $d$ has its own free parameters $\alpha$ and $\beta$, we define $f(d)$ as,
\begin{align}
  \label{eq:fSeriesDef}
  f(d) &= 1 + g \sum_{k=1}^\kmax \frac{1}{k!}\frac{d f^k}{d^k d} d^k + O(d^{\kmax+1})\\
  & = 1 + g \sum_{k=1}^\kmax A_k d^k + O(d^{\kmax+1})\\
\end{align}
where we define $A_k = \frac{1}{k!}\frac{d f^k}{d^k d}$ in order to emphasize the polynomial nature of our approximation to $f(d)$ and use $g$ to represent a random effect.
Here we assume $g \sim \text{Gamma}\left(\alpha_g, \beta_g = 1/\alpha_g\right)$ in order to ensure $\EE(g) = 1$, but other functions could be used.

Because $\phi$ and $A_1$ always co-occur, we cannot identify them separately from one another; as a result, we set $A_1 = 1$ and recognize that our estimates of $\phi$ are scaled relative to this term.
Using the results from \citet{Liang07} and \citet{Elphinstone85}, we can ensure that $f(d)$ is a monotonic, increasing function of $d$ by fitting our model using a transformation of variables $\alpha$ and $\beta$ and by restricting \kmax to multiples of 2. (Note that because $d > 0$, $f(d)$ is monotonic and increasing when $\kmax=1$.)
%Since we are assuming $A_0$ and $A_1 = 1$, when using the recurrence relation in Lang 2007, we set $\lambda = 1$ and $\alpha_1 = -1/2$.


\paragraph*{Defining Physiochemical Distances between Amino Acids :}
Assuming that functionality declines with an amino acid $a_i$'s physiochemical distance from the optimum amino acid \aopt at each site  provides a biologically defensible way of linking comparing genotypes that requires relatively few free parameters.
In addition, our approach naturally lends itself to model selection since we can compare the quality of our model fits using different mixtures of physiochemical properties.
Following \citet{Grantham74}, we focus on using composition $c$, polarity $p$, and molecular volume $v$ of each amino acid's side chain residue to define our distance function, but emphasize that other properties could be used.
We use the euclidian distance between residue properties where each property $c$, $p$, and $v$ has its own weighting term, $\alpha$, $\beta$, $\gamma$, respectively, [NOTE: WE MAY WANT TO USE $\alpha_c, \alpha_p, \ldots $ INSTEAD].
Because of similar identifiability issues we have with $A_1$ and $\phi$, we set $\gamma = 1$ and recognize that our our estimates of $\alpha$ and $\beta$ are scaled relative to $\gamma$.
More specifically,
\begin{equation*}
  d(a_i, \aopt) = \sqrt{\alpha \left(c\left(a_i\right) - c\left(\aopt\right)\right)^2 + \beta \left(p\left(a_i\right) - p\left(\aopt\right)\right)^2 +  \gamma \left(v\left(a_i\right) - v\left(\aopt\right)\right)^2}.
\end{equation*}



\subsubsection*{Linking Cost of Protein Synthesis to Allele Fixation}
In order to link the protein synthesis cost-benefit function $\eta$ of an allele with its fixation probability, we must make a number of assumptions.
First, we assume that each protein encoded within a genome carries out some beneficial function and that the organism needs that functionality to be produced at a target average rate $\phi$.
By definition, the optimal amino acid sequence for a given gene, \aoptvec, produces one unit of functionality.
Second, we assume that protein expression is regulated by the organism to ensure that functionality is produced at rate $\phi$.
As a result, the average protein production rate of a gene, $\psi$, is equal to $\phi/\Func(\avec)$ and the total energy flux allocated towards meeting the target functionality of a particular gene is $\eta(\cvec) \phi$. 
Third, we assume that every additional ATP spent per unit time to meet the organism's target function production rate $\phi$ leads to some slight proportional incremental decrease in fitness $W$.
This assumption, in turn, implies 
\begin{align}
  W_i\left(\cvec\right) &\propto \exp\left[- q \eta(\cveci) \phi\right].
\end{align}
where $q$ describes the decline in fitness with every ATP wasted per unit time $\phi$ and $\psi$ are measured in.
Correspondingly, the ratio of fitness between two genotypes is,
\begin{align*}
  W_i/W_j &=  \exp\left[- q \eta(\cveci) \phi\right]/\exp\left[- q \eta(\cvecj) \phi\right]\\
  &=  \exp\left[- q \left(\eta(\cveci)- \eta(\cvecj)\right) \phi\right]\\
\end{align*}
Given our assumptions about the \Cost and \Func functions above, this ratio simplifies to
\begin{align}
  W_i/W_j &\approx \left[- q \left(C_1 + C_2 n\right) \frac{1}{n}\left(\sum_{p \in \setP} \sum_{k=1}^\kmax A_k \left(d\left(\aip,\aoptp\right)^k - d\left(ajp,\aoptp\right)\right)\right) \phi \right]
%      &=  \exp\left[- q \left(C_1/n + C_2\right)\left(\sum_{p \in \setP} \sum_{k=1}^\kmax A_k \left(d\right(\aip,\aoptp\right)^k - d\left(ajp,\aoptp\right)^k\right) \phi\right]
\end{align}
where \setP represents the codon positions in which \cveci and \cvecj differ.
Fourth, we make a weak mutation assumption, such that alleles can differ at only one position at any given time, i.e.~$|\setP| = 1$, and that the population is evolving according to a Fisher-Wright model.
As a result, the probability a new mutant $j$ introduced via mutation into a resident population $i$ with effective size \Ne will go to fixation is,
\begin{align*}
  u_{i,j} &=  \frac{1 - \left(W_i/W_j\right)^b}{1 - \left(W_i/W_j\right)^\Ne}\\
   &\approx \frac{1-\exp\left[- b q \left(\eta(\cveci)- \eta(\cvecj)\right) \phi\right]}{1-\exp\left[- q \left(\eta(\cveci)- \eta(\cvecj)\right) \phi 2\Ne\right]}\\
   &= \frac{1- \exp\left[- b q \left(C_1/n + C_2\right)\left(\sum_{k=1}^\kmax A_k \left(d\right(\aip,\aoptp\right)^k - d\left(ajp,\aoptp\right)^k\right)\left) \phi\right]}{1-\exp\left[- q \left(C_1/n + C_2\right)\left(\sum_{k=1}^\kmax A_k \left(d\right(\aip,\aoptp\right)^k - d\left(ajp,\aoptp\right)^k\right)\left) \phi 2 \Ne\right]},
\end{align*}
where $b=1$ for a diploid population and $2$ for a haploid population \cite{Iwasa88,BergAndLassig03,SellaAndHirsh05}.
Finally, assuming a constant mutation between alleles $i$ and $j$, $\mu_{i,j}$, the transition rate from allele $i$ to $j$ can be modeled as,
\begin{align*}
  q_{i,j} = \begin{cases} 2 \mu_{i,j} \Ne /b & \text{if alleles $i$ and $j$ differ by one nucleotide}\\
    0 & \text{else}
  \end{cases}
\end{align*}
In the end, each optimal amino acid has a separte 64 x 64 substitution-rate matrix, which incorporates selection for the amino acid as well as the common mutation parameters across optimal amino acids. 
Thus, have overall 21 of such matrices, 20 for the amino acids, and one for stop codons.
Future work will allow transitions between optimal amino acids as well as between codons, which would result in a 21 x 64 = 1344 by 1344 matrix. 
In the meantime, however, amino acids represented by six codons, even if they are not all within one mutational step of each other, like Leucine (L), are assumed to share a single substitution-rate matrix.

While the overall model does not assume equilibrium, we need to scale the overall matrix in some way.
Traditionally, it is rescaled such that at equilibrium, one unit of branch length represents one expected substitution per site.
In our case, we want to do this scaling across all the matrices, since the branch lengths are used in common across the gene.
One wrinkle is that this must be done taking optimal amino acid frequency into account. 
Here the scaling is done jointly across all the 21 matrices to allow branch lengths under the fixed optimal amino acid model to be comparable to the branch lengths under the global model.
We calculate from the data a vector of 1344 empirical frequencies, \pi for each of the 64 codons observed when assuming each of 21 possible as the optimal amino acid (including stop codons).
A scaling factor is then calculated as the average rate -\sum_i{} \mu_ii*\pi_i=1, where $i$ indexes a particular codon for a particular optimal amino acid.
The final substitution-rate matrix is the original substitution-rate matrix multiplied by this scaling factor.
This matrix can then be applied to all the sites to calculate the likelihood. 

\subsubsection*{Likelihood calculation on a tree: }
Given our assumption of independent evolution among sites, the probability of the whole data set is the product of the probabilities of data at each individual sites.
Thus, the log likelihood is taken as 
\begin{align*}
put likelihood equation here
\end{align*}
The log likelihood is maximized by estimating the combined parameter for $C*q*\Phi$, Ne, two of the three Grantham distance parameters, \alpha_c, \alpha_p (again, we hold \alpha_gamma constant -- see above), the free mutation rate parameters (i.e., five free parameters if assuming GTR) and their three free nucleotide frequency parameters, \pi_i, given an alignment and a fixed tree topology.
We also assume each optimal acid across all sites as free parameters to be estimated in the model.
There are two ways in which we estimate the optimal amino acid at a given site: 1) use the majority rule of the amino acids observed across different species at a homologous site, and 2) numerically optimizing the choice of optimal amino acid at each site. 

In the case of including a random effect as described in Eq.(5) by specifying a discrete gamma, the log likelihood function becomes,
\begin{align*}
put gamma likelihood equation here
\end{align*}
where $k$ specifies the number of discrete categories.
Note that this would add an additional free parameter, \alpha_g, which describes the shape of the distribution.


\section*{Results}
\begin{enumerate}
\item Using \DeltaAIC as our measure, we see that even despite the need for estimating the optimal amino acid at each position in each protein, our model performs astronomically better than the standard GTR model.
\item In addition, we are able to generate estimates of gene expression which are well correlated with empirical estimates.
\item [Lots of other stuff]
\end{enumerate}
\section*{Discussion}

\begin{itemize}
\item Note that our definition of $\phi$ and our scaling of functionality differ slightly from our previous work \citep{Gilchrist07,GilchristEtAl09,ShahAndGilchrist11,GilchristEtAl15a}.
In our previous work, we were concerned with how changes in synonymous codons affected error rates and synthesis costs and, as a result, defined functionality relative to an error free protein, rather than an optimal one, and conflated $\phi$ and $\psi$.
\item Our approach requires relatively few parameter. 
  \begin{enumerate}
  \item Distance function $d(a_i, \aopt)$: If $n_d$ is the number of physiochemical properties examined, the number of parameters estimated is $n_d - 1$
  \item Benefit function $\Func$: If $n_A$ is the order of our Taylor Series approximation, the number of parameters is $n_A-1$.
  \item Gene expression $\phi$: One $\phi$ for each gene analyzed.
  \item Mutation bias: Depends on the model used it is either equal to the number of parameters in the model $n_\mu$ or $n_\mu-1$.
  \end{enumerate}
\item Our approach can be expanded by allowing the optimal amino acid to change during the course of evolution.
This should allow us to use a large, single matrix 400 x 400 matrix instead of 20 separate 20x20 matrices.
Further, if we may be able to compare the statistical properties of this extended transition matrix to the single transition matrix used in other approaches.
\item Statistical Physics model allows decoupling of \Ne, $\mu$, and strength selection.
\end{itemize}
\bibliographystyle{./am.nat}
\bibliography{./mike}

\subsubsection*{Notes for Jeremy}

\section*{Log}
\begin{itemize}
\item ``Defining the Benefit Function'' section began by mikeg on 7/23/15.
\item Methods expanded to include ``Defining Benefit Function'' on ?
\item {Linking Genotype Energetics to Fitness and Fixation} subsection added to Model.
\item Compiled on \today\xspace at \currenttime. 
\end{itemize}

\end{document}
